\chapter{模板使用}
\zhlipsum[1-2]
\section{图}
图~\ref{fig:39:f020390} 表明了。
\begin{figure}[!htb]
    \centering
    \includegraphics[width = 0.4\textwidth]{example-image}
    \caption{示例图。}
    \label{fig:39:f020390}
\end{figure}

\section{表}
表~\ref{tab:f20} 比较了结果。
\begin{table}[!htb]
    \centering
    \caption{不同参数的对比。}
    \begin{tabular}{cccccc}
        \toprule
         & $\omega_0$
         & $\gamma_0$
         & $\gamma_1$
         & $\kappa_1$
         & $\kappa_\mathrm{c}$
        \\
        \midrule
        我们的实验结果
         &
        1675~Hz
         &
        7.7~Hz
         &
        0.7~Hz
         &
        27.8~Hz
         &
        $\qty(-0.4+4.9\mathrm{i})$ Hz \\
        \cite{Zhong2020ReflectionAudioSounds}中的实验结果
         &
        1706~Hz
         &
        2.13~Hz
         &
        1.77~Hz
         &
        24~Hz
         &
        $\qty(-11+3.9\mathrm{i})$ Hz  \\
        \bottomrule
    \end{tabular}
    \label{tab:f20}
\end{table}


\section{公式}
式(\ref{eq:29jf2}) 是有编号的公式示例。
\begin{equation}
    f(x) = \int_0^x t\sin t \dd t
    \label{eq:29jf2}
\end{equation}

\section{引用}
引用示例\cite[图1]{Zhong2020InsertionLossThin}。

\section{列表}
\noindent {\bfseries \color{njuviolet}信号失真} \hfill {\color{njuviolet}\faCalendar*} 2024/9/1 -- 2024/12/1 \hfill {\color{njuviolet}\faUsers} 李梦同、吴旭祥
\begin{itemize}
    \item 实验对比不同调制算法的性能,包括 DSBAM、SSBAM、SRAM 等
    \item 使用 Volterra 滤波器进一步抑制非线性失真
\end{itemize}


\section{术语与英文缩写}
% 正文中使用示例
\gls{api} 是现代软件开发的核心组件。而 \gls{cpu} 和 \gls{gpu} 是计算机硬件的关键部件。
再次提到 \gls{api}、\gls{cpu} 和 \gls{gpu} 时,仅显示缩写。

\section{键盘按键}

LabShop开始测量(对应LabShop中按键 \key{F5})
\begin{lstlisting}[style=matlab-custom]
invoke(pulse, 'Start');
\end{lstlisting}

LabShop中按键 \key{Ctrl} + \key{3} 能打开某面板。