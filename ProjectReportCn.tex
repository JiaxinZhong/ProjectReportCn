\documentclass[cn]{ProjectReportCn}
% 使用注意:
%   - 编译工具:XeLaTeX;不支持 PDFLaTeX
%   - 参数 cn 表示中文,en 表示英文

%%%%%%%%%%%%%%%%%%%%%%%%%%%%%%%%%%%%%%%%%%%%%%%%%%%%%%%%%%%%%%%%%%%%%%%%%%%%%%%%
% 定义缩写,3个括号内分别是 索引名、缩写、全称
%    - 使用时,在正文中索引 \gls{acc}
%    - 全称仅在全文第一次引用时会显示,之后都只显示缩写
\newacronym{api}{API}{Application Programming Interface}
\newacronym{cpu}{CPU}{Central Processing Unit 中央处理器}
\newacronym{gpu}{GPU}{Graphics Processing Unit}
%%%%%%%%%%%%%%%%%%%%%%%%%%%%%%%%%%%%%%%%%%%%%%%%%%%%%%%%%%%%%%%%%%%%%%%%%%%%%%%%

%%%%%%%%%%%%%%%%%%%%%%%%%%%%%%%%%%%%%%%%%%%%%%%%%%%%%%%%%%%%%%%%%%%%%%%%%%%%%%%%
% 中英文章节标题处理
\ifthenelse{\equal{\lang}{en}}{
  \renewcommand{\chaptername}{Chapter}
}{
  \renewcommand{\chaptername}{第\,\thechapter\,章}
}

% \ifthenelse{\equal{\lang}{cn}}{
    % 自定义章节标题格式:Chapter n Title or 第n章 标题
    \makeatletter
    \def\@makechapterhead#1{%
      \vspace*{50\p@}%
      {\parindent \z@ \centering
        \normalfont\huge\bfseries
        \ifthenelse{\equal{\lang}{en}}{
          \chaptername\ \thechapter\quad #1
        }{
          \chaptername\quad #1
        }
        \par\nobreak
        \vskip 40\p@
      }}
    \makeatother
% }{
    % \titleformat{\chapter}[hang]{\bfseries\large}{\thechapter}{1em}{\bfseries}
    % % 调整 chapter 与上下内容的间距
    % \titlespacing*{\chapter}{0pt}{-10pt}{10pt}  % 左缩进,上间距,下间距
    % \titleformat{\section}[hang]{\bfseries\normalsize}{\thesection}{1em}{\bfseries}
    % \titleformat{\subsection}[hang]{\bfseries\normalsize}{\thesubsection}{1em}{\bfseries}
% }
%%%%%%%%%%%%%%%%%%%%%%%%%%%%%%%%%%%%%%%%%%%%%%%%%%%%%%%%%%%%%%%%%%%%%%%%%%%%%%%%

%%%%%%%%%%%%%%%%%%%%%%%%%%%%%%%%%%%%%%%%%%%%%%%%%%%%%%%%%%%%%%%%%%%%%%%%%%%%%%%%
% setting for levels
\usepackage{hyperref}
\setcounter{secnumdepth}{2}
\setcounter{tocdepth}{2}
\hypersetup{
    bookmarksopenlevel = 3,
	colorlinks		=	true,
	bookmarks		=	true,
	bookmarksopen	=	true,
	pdfstartview	=	Fit,
	pdftitle		=	{项目报告模板},
	pdfauthor		=	{钟家鑫}, 
}
%%%%%%%%%%%%%%%%%%%%%%%%%%%%%%%%%%%%%%%%%%%%%%%%%%%%%%%%%%%%%%%%%%%%%%%%%%%%%%%%

%%%%%%%%%%%%%%%%%%%%%%%%%%%%%%%%%%%%%%%%%%%%%%%%%%%%%%%%%%%%%%%%%%%%%%%%%%%%%%%%
%% 以下是使用 bibtex 
% \bibliographystyle{unsrt}

%% 以下是使用 biblatex
% biber 需要编译: xelatex -> biber -> xelatex*2
\usepackage[
    giveninits=true, 
    maxbibnames=99,
    % style=numeric,
    sorting=none,
    doi = false,
    url = false,
    isbn = false,
    % style=ieee, 
    % citestyle=numeric-comp,
    style=gb7714-2015,
    ]{biblatex}
\addbibresource{biblatex.bib}
%%%%%%%%%%%%%%%%%%%%%%%%%%%%%%%%%%%%%%%%%%%%%%%%%%%%%%%%%%%%%%%%%%%%%%%%%%%%%%%%

%%%%%%%%%%%%%%%%%%%%%%%%%%%%%%%%%%%%%%%%%%%%%%%%%%%%%%%%%%%%%%%%%%%%%%%%%%%%%%%%
% 语言处理
\ifthenelse{\equal{\lang}{cn}}{
    \renewcommand{\contentsname}{目录}
    \renewcommand{\listfigurename}{插图目录}
    \renewcommand{\listtablename}{表格目录}
    \renewcommand{\acronymname}{缩略语}
    \renewcommand{\glossaryname}{符号说明}
    \renewcommand{\appendixname}{附录}
    \renewcommand{\bibname}{参考文献} % 用于 book 类
    \renewcommand{\refname}{参考文献} % 用于 article 类
}{
    \renewcommand{\contentsname}{Table of Contents}
    \renewcommand{\listfigurename}{List of Figures}
    \renewcommand{\listtablename}{List of Tables}
    \renewcommand{\acronymname}{List of Abbreviations}
    \renewcommand{\glossaryname}{List of Symbols}
    \renewcommand{\appendixname}{Appendix}
    \renewcommand{\bibname}{References}
    \renewcommand{\refname}{References}
}
%%%%%%%%%%%%%%%%%%%%%%%%%%%%%%%%%%%%%%%%%%%%%%%%%%%%%%%%%%%%%%%%%%%%%%%%%%%%%%%%

\renewcommand{\chaptermark}[1]{%
    \markboth{\thechapter. #1}{}}

% 报告的标题
\ifthenelse{\equal{\lang}{cn}}{
    \def\titlestr{中文项目报告模板}
}{
    \def\titlestr{English Project Report}
}

%%%%%%%%%%%%%%%%%%%%%%%%%%%%%%%%%%%%%%%%%%%%%%%%%%%%%%%%%%%%%%%%%%%%%%%%%%%%%%%%
% 报告的日期
\usepackage{datetime2}  % 更灵活的日期处理
\ifthenelse{\equal{\lang}{cn}}{
    % \renewcommand{\dateseparator}{年}
    \newcommand{\datestr}{\the\year 年 \number\month 月 \number\day 日}
}{
    \newcommand{\datestr}{\number\year-\number\month-\number\day}
}
\date{\datestr}
%%%%%%%%%%%%%%%%%%%%%%%%%%%%%%%%%%%%%%%%%%%%%%%%%%%%%%%%%%%%%%%%%%%%%%%%%%%%%%%%

\begin{document}
%%%%%%%%%%%%%%%%%%%%%%%%%%%%%%%%%%%%%%%%%%%%%%%%%%%%%%%%%%%%%%%%%%%%%%%%%%%%%%%%
%% 封面制作
\pdfbookmark{封面}{cover}
% \maketitle
\makeatletter
\begin{titlepage}
    \centering
    \vspace{1cm}
    % 南京大学Logo
    \begin{tikzpicture}
        \node (image) at (0,0) {\njuemblem{!}{4cm}};

        \node (image) at (5.5cm, .5cm) {\njuname{5cm}{!}};

        \node (image) at (5.5cm, -1cm) {\njuname*{7cm}{!}};

    \end{tikzpicture}

    \vspace{-.5cm}
    {\LARGE \bfseries 声学研究所\\[5mm] 噪声控制与通信声学实验室}

    \ifthenelse{\equal{\lang}{en}}{
        {\LARGE \bfseries Institute of Acoustics\\[5mm] Audio Acoustics Laboratory}
    }

    \vspace{4cm}

    { \fontsize{36}{72}\selectfont \bfseries \titlestr}

    \vspace{.5cm}
    {\bfseries \Large \@date}

    \vfill
    \noindent
    \begin{minipage}[t]{.5\textwidth}
        \raggedright
        \doublespacing
        \ifthenelse{\equal{\lang}{cn}}{
            {\LARGE \textbf{呈送}}\\
            {\large 某某某}\\
            {\large 某某技术有限公司}\\
            {\large 地址}\\
            {\large 电话: 0000-0000000}\\
            {\large Email: foo@xxx.com}\\
        }{
            {\LARGE \textbf{Submitted to}}\\
            {\large Foo Name}\\
            {\large ABC Technology Co., Ltd.}\\
            {\large Address}\\
            {\large Tel: 0000-0000000}\\
            {\large Email: foo@xxx.com}\\
        }
    \end{minipage}% 
    \begin{minipage}[t]{.5\textwidth}
        \raggedleft
        \doublespacing
        \ifthenelse{\equal{\lang}{cn}}{
            {\LARGE \textbf{起草}}\\
            {\large 某某某、某某某}\\
            {\large 南京大学声学研究所}\\
            {\large 南京 210093 中国}\\
            {\large 电话: 0000-0000000}\\
            {\large Email: foo@xxx.com}\\
        }{
            {\LARGE \textbf{Prepared by}}\\
            {\large Jia-Xin Zhong, et al.}\\
            {\large Institute of Acoustics, Nanjing University}\\
            {\large Nanjing 210093, China}\\
            {\large Tel: 0000-0000000}\\
            {\large Email: foo@xxx.com}\\
        }
    \end{minipage}
\end{titlepage}
\makeatother
%%%%%%%%%%%%%%%%%%%%%%%%%%%%%%%%%%%%%%%%%%%%%%%%%%%%%%%%%%%%%%%%%%%%%%%%%%%%%%%%

%% 目录
\pdfbookmark{目录}{contents}
\ifthenelse{\equal{\lang}{cn}}{
    \renewcommand{\contentsname}{目录}
}{
    \renewcommand{\contentsname}{Table of Contents}
}
\tableofcontents

% 使用 第x页 | 共xx页的样式
\thispagestyle{firststyle}

%%%%%%%%%%%%%%%%%%%%%%%%%%%%%%%%%%%%%%%%%%%%%%%%%%%%%%%%%%%%%%%%%%%%%%%%%%%%%%%%
% 生成图索引页(单独成页,注释掉可取消)
\clearpage
% \renewcommand{\listfigurename}{图索引} % 标题
\phantomsection % 配合 hyperref 定位
\addcontentsline{toc}{chapter}{\listfigurename} % 将索引页加入目录(可选)
\listoffigures
% \thispagestyle{empty} % 取消索引页页眉页脚(可选)
%%%%%%%%%%%%%%%%%%%%%%%%%%%%%%%%%%%%%%%%%%%%%%%%%%%%%%%%%%%%%%%%%%%%%%%%%%%%%%%%


%%%%%%%%%%%%%%%%%%%%%%%%%%%%%%%%%%%%%%%%%%%%%%%%%%%%%%%%%%%%%%%%%%%%%%%%%%%%%%%%
% 生成表索引页(单独成页)
\clearpage
% \renewcommand{\listtablename}{\listtablename}
\phantomsection
\addcontentsline{toc}{chapter}{\listtablename}
\listoftables
% \thispagestyle{empty}
%%%%%%%%%%%%%%%%%%%%%%%%%%%%%%%%%%%%%%%%%%%%%%%%%%%%%%%%%%%%%%%%%%%%%%%%%%%%%%%%

%%%%%%%%%%%%%%%%%%%%%%%%%%%%%%%%%%%%%%%%%%%%%%%%%%%%%%%%%%%%%%%%%%%%%%%%%%%%%%%%
\clearpage
\phantomsection
\addcontentsline{toc}{chapter}{\acronymname} % 加入目录(chapter层级)
\printglossary[type=\acronymtype, title={\acronymname}, nonumberlist=false] % 显示页码
%%%%%%%%%%%%%%%%%%%%%%%%%%%%%%%%%%%%%%%%%%%%%%%%%%%%%%%%%%%%%%%%%%%%%%%%%%%%%%%%


\ifthenelse{\equal{\lang}{en}}{
    \def\AbstractName{Abstract}
}{
    \def\AbstractName{摘要}
}
\chapter*{\AbstractName}
\addcontentsline{toc}{chapter}{\AbstractName}
这里撰写摘要内容,对全文进行总结,特别是读者比较关心的内容和数据。

如果有项目验收指标,需要在这里总结项目方关心的数据,并针对性描述是否达到验收标准。
例如:总谐波失真小于 10 \%,互调失真小于30 \%,满足项目验收目标。

\lipsum[1]
% \zhlipsum[1-3]

\chapter{简介}
中文的项目报告模板。
请使用 \XeLaTeX 进行编译。
% 建议使用 \texttt{pdflatex} 进行编译。
建议用不同的文件来写不同章,然后在主文件中使用 \lstinline!\include! 汇总。

\chapter{模板使用}
\zhlipsum[1-2]
\section{图}
图~\ref{fig:39:f020390} 表明了。
\begin{figure}[!htb]
    \centering
    \includegraphics[width = 0.4\textwidth]{example-image}
    \caption{示例图。}
    \label{fig:39:f020390}
\end{figure}

\section{表}
表~\ref{tab:f20} 比较了结果。
\begin{table}[!htb]
    \centering
    \caption{不同参数的对比。}
    \begin{tabular}{cccccc}
        \toprule
         & $\omega_0$
         & $\gamma_0$
         & $\gamma_1$
         & $\kappa_1$
         & $\kappa_\mathrm{c}$
        \\
        \midrule
        我们的实验结果
         &
        1675~Hz
         &
        7.7~Hz
         &
        0.7~Hz
         &
        27.8~Hz
         &
        $\qty(-0.4+4.9\mathrm{i})$ Hz \\
        \cite{Zhong2020ReflectionAudioSounds}中的实验结果
         &
        1706~Hz
         &
        2.13~Hz
         &
        1.77~Hz
         &
        24~Hz
         &
        $\qty(-11+3.9\mathrm{i})$ Hz  \\
        \bottomrule
    \end{tabular}
    \label{tab:f20}
\end{table}


\section{公式}
式(\ref{eq:29jf2}) 是有编号的公式示例。
\begin{equation}
    f(x) = \int_0^x t\sin t \dd t
    \label{eq:29jf2}
\end{equation}

\section{引用}
引用示例\cite[图1]{Zhong2020InsertionLossThin}。

\section{列表}
\noindent {\bfseries \color{njuviolet}信号失真} \hfill {\color{njuviolet}\faCalendar*} 2024/9/1 -- 2024/12/1 \hfill {\color{njuviolet}\faUsers} 李梦同、吴旭祥
\begin{itemize}
    \item 实验对比不同调制算法的性能,包括 DSBAM、SSBAM、SRAM 等
    \item 使用 Volterra 滤波器进一步抑制非线性失真
\end{itemize}


\section{术语与英文缩写}
% 正文中使用示例
\gls{api} 是现代软件开发的核心组件。而 \gls{cpu} 和 \gls{gpu} 是计算机硬件的关键部件。
再次提到 \gls{api}、\gls{cpu} 和 \gls{gpu} 时,仅显示缩写。

\section{键盘按键}

LabShop开始测量(对应LabShop中按键 \key{F5})
\begin{lstlisting}[style=matlab-custom]
invoke(pulse, 'Start');
\end{lstlisting}

LabShop中按键 \key{Ctrl} + \key{3} 能打开某面板。
\chapter{更新记录}

\section{初版}
\begin{itemize}
    \item 制作人:钟家鑫
    \item 时间:\today
\end{itemize}


%%%%%%%%%%%%%%%%%%%%%%%%%%%%%%%%%%%%%%%%%%%%%%%%%%%%%%%%%%%%%%%%%%%%%%%%%%%%%%%%
% 附录区域
%%%%%%%%%%%%%%%%%%%%%%%%%%%%%%%%%%%%%%%%%%%%%%%%%%%%%%%%%%%%%%%%%%%%%%%%%%%%%%%%
\begin{appendices}
    % 标题风格
    \titleformat{\chapter}{\centering\huge\bfseries}{\appendixname\,\thechapter\,}{1em}{}

    % 下面插入相应的 tex 文件
    \chapter{  式 (\ref{eq:29jf2}) 的推导}

\zhlipsum[1]
\section{问题1}
\zhlipsum[2]
\section{问题2}
\zhlipsum[3]
    \chapter{代码}

\zhlipsum[2-3]
\section{大气中声吸收系数的计算}
MATLAB 和 Python 代码分别见代码\,\ref{code:matlab:absorp}\,和\,\ref{code:python:absorp}\,所示。
提交报告时,请将有关代码整体存到 \codeinline{code/} 文件夹下。

\lstinputlisting[
    style=matlab-custom, 
    caption=\codecaptionwithlink{吸声系数计算}{code/matlab/AbsorpAttenCoef.m},
    label=code:matlab:absorp
]{
    code/matlab/AbsorpAttenCoef.m
}

\lstinputlisting[
    style=python-custom, 
    caption=\codecaptionwithlink{吸声系数计算}{code/python/AbsorpAttenCoef.py},
    label=code:python:absorp
]{
    code/python/AbsorpAttenCoef.py
}
\end{appendices}


%% biblatex
\printbibliography[heading=bibintoc, title=\bibname]

%% bibtex
% \bibliography{bibtex}

\end{document}

